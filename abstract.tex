\glspl{MSR} are advanced reactors noted for strong passive safety features.
They present unique challenges in multiphysics reactor modeling \& simulation arising from
strong temperature reactivity feedback, delayed neutron precursor flow, and turbulent heat transport
in fuel salt regions. Simulating the complex multiphysics interactions in
\glspl{MSR} requires robust, flexible, and highly scalable multiphysics software. Many \gls{MSR}
designs also still retain control rods which can complicate time-dependent reactivity-initiated
transient simulations on reactor software relying on neutron diffusion theory.
This work builds on existing capabilities in Moltres, a MOOSE-based \gls{MSR} simulation software,
to tackle these challenges and support efforts towards \gls{MSR} deployment.

This work verified and validated existing multiphysics coupling capabilities in Moltres in two
comparative studies. In both studies, Moltres showed good agreement with other \gls{MSR} simulation tools
involving coupled neutronics and thermal-hydraulics problems. This work also introduces a
turbulence model in Moltres to support future \gls{MSR} analyses involving turbulent delayed
neutron precursor and temperature transport. Lastly, this work introduces a novel hybrid
$S_N$-diffusion method for accurate control rod modeling in time-dependent \gls{MSR} simulations.
The hybrid method combines the strengths of both approaches by generating transport corrections
using the $S_N$ method near control rods. The $S_N$ and neutron diffusion solvers are coupled
through an adaptive boundary coupling algorithm. This algorithm allows the solver to adapt to the
transport correction parameters and preserve smooth neutron flux gradients across the interface.
In 1-D, 2-D, and 3-D $k$-eigenvalue simulations, the hybrid method produced accurate control
rod worth estimates, relative to reference neutron transport solutions and experimental data,
at approximately four times the cost of the neutron diffusion method.
Demonstrations of the hybrid method for time-dependent rod drop and reactivity insertion
simulations also perfomed well in reproducing expected trends observed in experimental data.
Analysis of its computational performance indicated the possibility of further optimizations
beyond its current implementation. With the hybrid method's spatial resolution and
efficient computational performance, Moltres could enable accurate and cost-effective simulations
of asymmetric transient in \glspl{MSR}.
